\subsection{Farm Model System}


\subsubsection{Spoofing}
\par\hfill
\par  Um atacante pode fazer-se passar por um sensor e transmitir dados falsos para o gateway 1), o mesmo pode acontecer se o atacante tentar fazer-se passar pelo gateway e enviar dados para a cloud 3) ou caso se faça passar pela cloud e envie dados errados ao gateway 4). Este ataques pode ser facilmente evitado usando mecanismos que permitam identificar inequivocamente cada elemento do sistema.\newline

\subsubsection{Tampering}
\par\hfill
\par Um atacante pode alterar informação dos dados recolhidos pelo gateway através dos sensores. Para evitar que este cenário se verifique basta ter ACLs que limitem de forma correta as atividades dos utilizadores.


\subsubsection{Repudiation}
\par\hfill
\par Um atacante pode alterar dados armazanados no gateway e apagar o log para que não se saiba quem os alterou. Para garantir que se salvaguarda o log basta guardar os registos de acesso ao gateway numa máquina á parte na qual se mantenha uma cópia do log e só se permita escrever, como na cloud por exemplo.

\subsubsection{Information Disclosure}
\par\hfill
\par Caso a informação não seja encriptada, esta pode ser lida por um concorrente que não pague o serviço dispendioso e que possa obter vantagem competitiva simplesmente ao intercetar as informações enviadas pelos sensores 1). O mesmo pode acontecer quando a informação é enviada pelo gateway á cloud processo 3) e pela cloud ao gateway 4). Esta falha pode ser colmatada fazendo uso de encriptação com chave simétrica conhecida apenas entre os dispositivos do sistema, ou usando mecanismos de chave publica.\newline

\subsubsection{Denial of Service}
\par\hfill
\par Um atacante pode tentar levar a cabo um ataque DoS nos processos 1), 3) e 4), gerando um monte de pedidos aos quais o sistema não consiga responder. No caso de tentativa de Denial of Service podemos programar os nosso equipamentos para, após um X número de tentativas, bloquear o IP (temporáriamente) de quem está a gastar recursos do sistema sem se autenticar.\newline

\textit{Nota:} No caso da Cloud, está a cargo do provedor do serviço tratar casos de tentativas de Denial of Service.\newline



\textit{Nota:} Em relação ao processo 2), como a ligação é feita usualmente por cabo não existem problemas muito serios, apenas ataques fisicos devem ser considerados, como cortar os cabos de ligação para impedir o funcionamento do sistema.\newline


\section{Considerações Finais}

\subsection{Front-End}
\hfill

A segurança necessária para assegurar o subsistema do Front-End baseia-se fortemente na encriptação da informaçao que é enviada e recebida pelos utilizadores/servidor(HTTPS,IPsec) como também pela utilização de Assinaturas e MACs para garantir a integridade. Finalmente também dever-se-á utilizar ACLs e Profiling de Tráfego para previnir acesso indevidos e DoS ,respetivamente.


\subsection{Sensores}
\par\hfill
\par Como visto anteriormente os sensores devem ser autenticado inequivicamente e enviar a informação recolhida encriptada para garantir a veracidade e origem dos dados.\newline
\par Para garantir que um atacante não adiciona um sensor com código modificado de forma a de alguma forma comprometer o sistema, os sensores a poderem comunicar com o gateway devem ser definidos á partida.\newline

\subsubsection{ZigBee sensors}
\begin{itemize}
\item Quanto á identificação dos mesmos esta é feita de forma inequivoca, garantindo que não é possível um atacante replicar o id do sensor~\cite{ref_url1}.\newline 
\item Usa AES-128 encryption para garantir a encriptação dos dados~\cite{ref_url1}.
\end{itemize}
\subsubsection{TelosB motes}
\begin{itemize}
\item Usa TinyOS como a biblioteca criptográfica normalmente, esta tem um algoritmo eficiente com overhead de encriptação na ordem dos microsegundos, garantindo a integridade do sistema de forma rápida e utilizando poucos recursos~\cite{ref_url2}.\newline
\item Relativamente barato~\cite{ref_url2}.
\end{itemize}
\subsubsection{Arduino}
\begin{itemize}
\item Normalmente usa AES-128 encryption para garantir a encriptação dos dados~\cite{ref_url3.1}~\cite{ref_url3.2}.\newline
\par Usa X.509 que é um padrão ITU-T para infraestruturas de chaves públicas (ICP),permitindo autenticação forte~\cite{ref_url3.3}~\cite{ref_url3.4}.
\item Permite programar os processos de autenticação dos sensores e como estes comunicam. 
\end{itemize}
\subsubsection{Raspberry}
\begin{itemize}
\item Usa vários mecanismos para garantir a encriptação, como RC4 encryption,  entre outros.~\cite{ref_url4.1}.
\item Relativamente barato~\cite{ref_url4.1}.\newline
\end{itemize}

\par Concluindo, se o foco for ter uma encriptação o mais forte possível, devemos optar por utilizar um raspberry sensor visto este usar mecanismos de chave pública. No entanto estes pesam no requisito de transmissão de dados de forma mais rápida possível. O TelosB motes tem essa vantagem embora não possua uma encriptação tão boa como os restantes sensores.\newline


\textit{Nota:} Se se pretende uma tecnologia de encriptação que se mantenha segura sem ser necessário mudá-la, o AES-128 não é aconselhado visto que já está em fase de se tornar deprecated (estima-se que por volta de 2030) e a quinta supõe-se que deva permanecer em funcionamento por mais de 10/11 anos aquando da redação deste relatório.\newline



\subsection{Gateway}
\par\hfill
\par No caso da informação que flui do gateway para a cloud é preciso garantir que estamos a enviar dados encriptados visto que o provedor da cloud pode utilizar os dados que registamos para proveito próprio, ou até mesmo um terceiro que comprometa a segurança da cloud não possa tirar proveito da nossa informação. Desta forma temos de garantir que apenas o nosso gateway e a aplicação no lado do cliente tenha acesso á chave de encriptação e desencriptação. Note-se que se deve fazer uso de encriptação forte para garantir de forma mais consistente a segurança dos dados.\newline

\newpage
\subsection{Cloud}
\par\hfill
\par O tratamento da segurança da informação na cloud (encriptada ou não) é responsabilidade do provedor do serviço, logo este relatório não englobará medidas para proteger a cloud de qualquer ataque. Apenas de proteger a informação que lhe é passada encriptando-a.\newline
Para além disso é necessário ainda assim que tenhamos uma forma inequivoca de garantir o não repúdio por parte do gateway face a cloud e vice-versa, nesta matéria aconselha-se o uso de um certificado digital.\newline 



